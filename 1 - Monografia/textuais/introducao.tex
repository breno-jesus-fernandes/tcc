
\section{Introdução}

\textcolor{gray}{Trazer algo pra interessar o leitor. A introdução estabelece uma \colorbox{lightgray}{contextualização fundamentada do tema}, assim como apresenta o problema de pesquisa (questão a ser respondida pelo TCC) \textcolor{black}{[entregue no Projeto Parcial]}, o objetivo geral (meta a ser alcançada, sendo sua apresentação feita com verbo no infinitivo) \textcolor{black}{[entregue no Projeto Parcial]} e os objetivos específicos (etapas de trabalho para atingir o objetivo geral \textcolor{black}{[entregue no Projeto Parcial]}, também apresentados com verbos no infinitivo dado que traduzem ações). Além disso, trata da \colorbox{lightgray}{justificativa} (elementos e referências bibliográficas que mostram a relevância da investigação, assim como contribuições decorrentes do trabalho proposto).}
\textcolor{gray}{Finalmente, ainda são dispostas a hipótese (afirmação provisória a ser confirmada/refutada durante o TCC) [entregue no Projeto Parcial] e a \colorbox{lightgray}{organização do documento} (descrição das partes constitutivas).}