\section{Materiais e Métodos}
\textcolor{gray}{Apresentação das etapas e dos materiais (hardwares/softwares etc.) que serão utilizados ao longo da pesquisa, assim como as justificativas pertinentes (embasadas por referências bibliográficas). Inclusive, também é necessário \colorbox{lightgray}{classificar a pesquisa} conforme a \colorbox{lightgray}{natureza} (básica ou aplicada), \colorbox{lightgray}{abordagem} (quantitativa, qualitativa, mista ou quali-quanti), \colorbox{lightgray}{finalidade} (exploratória, descritiva, explicativa, metodológica) e \colorbox{lightgray}{meio} (bibliográfica, documental, estudo de caso, experimental, pesquisa de campo etc.).}
\textcolor{gray}{Cabe destacar que as imagens presentes no documento devem possuir, conforme exemplo da Figura 1, a resolução de 600dpi.} 
\begin{figure}[h]
\centering

\caption{Marca da FCI-UPM}
\end{figure}

\textcolor{gray}{Já a listagem das referências completas que serão estudadas ao longo da pesquisa deve ser apresentada no final do documento (sendo as informações apresentadas no arquivo ``\texttt{estilo/sbc-template.bib}''. Ao longo do texto, as referências fazem uso do comando  \textbackslash\texttt{cite\{chamada-da-referência\}}, resultando, por exemplo, em \cite{boulic:91}, \cite{smith:99} e \cite{knuth:84}.}
